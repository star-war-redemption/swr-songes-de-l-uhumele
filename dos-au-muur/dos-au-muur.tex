\documentclass{jdrp}

\bibliography{dos-au-muur/references} 

\newcommand*{\crg}{{\aurebesh\Large \$}} % Symbol for Galactic Credits

\hypersetup{
	pdftitle={SWR - Dos au Muur},
	pdfsubject={Scénario, Dos au Muur},
	pdfauthor={Marthym},
	pdfkeywords={starwars,savage,worlds,jdr,scenario},
	pdfcopyright={This work is licensed under the Creative Commons Attribution-ShareAlike 4.0 International License.}
}

\begin{document}

	\begin{titlepage}

	\begin{center}
		\hspace*{\vfill}
		\noindent\Huge\jedifont{Star Wars Redemption}\\ 
		\noindent\fontsize{50}{70}\jedifont{\$}
		\noindent\fontsize{50}{70}\jedifont{\#}\\
		\noindent\fontsize{50}{60}\jedifont{Dos au Muur}
		\hspace*{\vfill}
	\end{center}

	%\hspace*{\vfill}

	\noindent\makebox[\textwidth]{
		\includegraphics[width=\paperwidth]{_img/cover-bg.png}}
	\begin{tikzpicture}[overlay]
		\node[minimum width=180pt,minimum height=180pt, rotate=30] at (15,11){\includegraphics[width=180pt]{_img/dos-au-muur/talisman.png}};
	\end{tikzpicture}}
	\end{titlepage}

	\onecolumn
	\section{contexte de campagne}
	
	\begin{wrapfigure}{R}{180pt}
		\centering
		\includegraphics[width=180pt]{_img/dos-au-muur/talisman.png}
		\caption{\label{fig:talisman-de-muur}Talisman de Muur}
		\vspace{1\baselineskip}
		\includegraphics[width=180pt]{_img/dos-au-muur/karness-muur.png}
		\caption{\label{fig:karness-muur}Karness Muur}
	\end{wrapfigure}
	
	Cette campagne est écrite initialement pour \citetitle{jdrp-starwars} mais un scénar reste un scénar et il est jouable dans n’importe quel univers de Star Wars.

	L’idée était de faire une campagne d’introduction avec des personnages partant de rien. Les joueurs commencent Novice et n’ont pas besoin d’historique complexe et élaboré (bien que cela ne soit pas interdit bien sûr). De cette façon, les personnages devraient être assez vite fait. Et elle est adaptable aussi bien avec des joueurs orienté Alliance Rebelle que Empire. Dans les deux cas, l’objectif sera le même mais les dessains changeront.

	La campagne se déroule dans les premières années de l’avènement de l’Empire, au MJ de voir s’il veut préciser.

	La trame de la campagne se base sur un très ancien artéfact Sith, le \citetitle{talisman-de-muur}. Un artéfact créé par Karness Muur, un Sith se servant de la Force pour prolonger sa vie. L’artéfact contient l’âme de Muur, celui qui le porte est possédé par Karness et peut contrôler les Rakghoules.


	On trouve beaucoup d’informations sur cet artéfact sur HoloNet et je me suis grandement inspiré de ces informations pour cette campagne en faisant vivre à mes héros les aventures des divers individus qui ont croisé le Talisman\ldots

	\twocolumn

	\section{Sauvetage du Pelican}

Voici un premier scénario pour la campagne \textbf{Dos au Muur}. 

\vspace{2\baselineskip}
\begin{flushright}
	\emph{Pelican (IA-1701)}
\end{flushright}

\vspace{-6\baselineskip}

\hspace{-0.4\columnsep}\includegraphics[width=\textwidth]{_img/dos-au-muur/venator.png}
\vspace{-7\baselineskip}

\subsection{La rencontre}
Nos héros ne se connaissent pas encore. Ils ont répondu à un contract proposé par \emph{Industrial Automaton} pour une mission de récupération.

Ils ont rendez-vous sur l’avant-poste commercial de l’anneau de Kafrene au Starlord Café. \'A leur arrivée, ils sont conviés dans une arrière salle ou les attend \textbf{Vyna Anen} un Sluissi, le représentant de IA. En plus du groupe de nos héros, deux autres personnes ont répondu au contract, un Abyssin et un Rodien.

\begin{quote}
	Messieurs bonjour, je représente Indrustrial Automaton.
	Comme vous avez peu le voir sur le contract auquel vous avez répondu, nous cherchons à rassembler une équipe pour récupérer l’un de nos prototypes de droïde Type R perdu sur un Croiseur dont nous n’avons plus de nouvelle.
	Le \textbf{Pelican IA-1701} n’a plus donné signe de vie depuis 10h et 33mn maintenant. Il avait à son bord le seul prototype de notre dernier Type R. Il est vital pour nous de récupérer ce prototype intact.

	Si vous acceptez la mission, une navette droïde vous conduira directement à la dernière position connue du Pelican. Cette mission doit resté confidentielle, nous ne tenons pas à ce que le public sache qu’Industrial Automaton perde ses vaisseaux !
	En cas de succés la somme convenue sera virée directement sur vos comptes respectifs. Dans le cas contraire vous serez mort ou en passe de l’être.

	Y a-t’il des questions ?

	\ldots

	Bien, la navette décollera de l’astroport, quai N°5 dans une heure, elle n’attendra personne.
\end{quote}

Voilà qui donne le ton et la direction du scénario. Les héros disposent donc d’une heure, s’ils le souhaitent pour faire quelques emplettes puis direction la navette. Si des joueurs demandent à prendre leur propre vaisseau, cela est impossible, l'emplacement du vaisseau doit rester confidentiel.

\subsection{Pelican Bay}
\'A la sortie d’hyperespace, les héros trouvent croiseur de classe Venator en bon état mais à la dérive. La navette tente une procédure d’appontage mais le système de guidage ne semble pas fonctionner, le droïde aux commandes de la navette ne cesse de répéter :
\begin{flushright}
	Erreur trop bas tros bas !\\
	Erreur trop bas tros bas !\\
	\ldots
\end{flushright}

\vspace{3\baselineskip}
\begin{quote}
	Erreur trop bas tros bas !\\
	Erreur trop bas tros bas !\\
	Erreur trop bas tros bas !\ldots
\end{quote}

Et la navette s’échoue lamentablement dans le hangar. Comme de par hasard, et pour bien appuyer sur la gravité de la situation, l’Abyssin et le Rodien sont mort pendant le crash et le droïde pilote est dans un sale état. Il ne pourra donner aucune information.

Une fois à bord du Pelican, une alarme est en cours et une voie pré-enregistrée se fait entendre à intervalles réguliers :

\begin{quote}
	Alerte trajectoire, le vaisseau se dirige actuellement vers une étoile, point de non-retour dans 2h53mn ! 
	Voyez modifier la trajectoire du vaisseau !\\ 
	\ldots
\end{quote}

\'A l’intérieur du vaisseau c’est la désolation, des cadavres partout, du sang sur les murs, des traces de griffures sur les murs. L’éclairage est partiellement en panne les néons scintillent, des débrits entravent la marche des héros. Leur champ de vision est réduit à cause des conditions à bord. Malgré tout ça, les systèmes de survie et la gravité artificielle fonctionnent.\\

Le bruit du crash de la navette a attiré la population locale, à savoir les Rakghouls~\ref{sec:rakghoul} (p. \pageref{sec:rakghoul}). Lancer 1d4 pour savoir combien se pointent. Si vous n’avez que 2 ou trois héros sous la main, ajustez avec 1d4-1.

\noindent\includegraphics[width=\linewidth]{_img/dos-au-muur/rakghoul.png}

Le but des héros devrait maintenant être double, trouver le fameux prototype mais aussi trouver un moyen de sauver leur peau. 

\subsection{Exploration}
\emph{Numérotez les pièces de votre vaisseau et lancez un dé pour savoir dans quelle pièce se trouve le droïde.}\\

Une fois la première vague de Rakghouls balayé, nos héros se retrouvent dans le hangar, la porte de se dernier est ouverte.

Un jet de Recherche (-2 pour l’obscurité) réussi permet de trouver un kit de survie avec un Medipac, une lampe torche et des rations de survie.

Pas loin de la porte du hangar se trouve un terminal dont l’écran clignote en rouge au rythme de l’alarme. Les héros peuvent tenter des jets de Piratage pour les actions suivantes :
\begin{rebelist}
	\item Stopper l’alarme. 
	\item Obtenir un plan du vaisseau. En cas de succès, le plan s’affiche mais le terminal rend l’âme au bout de quelques secondes et le plan s’efface. En cas de relance le plan reste affiché et le terminal est utilisable.
	\item Obtenir l’inventaire des navettes présente dans les hangars du vaisseau. Un succès leur apprend qu’il y en a une dans le hangar du niveau inférieur. Une Relance leur apprend que la navette est dans le hangar 4-2.
\end{rebelist}
Dans tous les cas, un échec critique fait disjoncter le terminal définitivement.

Les héros avancent donc dans les couloirs du Pelican à la recherche du droïde et d’un moyen de dévier le vaisseau. Faites faire un jet de Discrétion de temps à autre. En cas d’échec, 2 Rakghouls se pointent (plus si vous voyez que vos héros s’en tire trop facilement).

Au fur et mesure que les aventuriers avancent, utilisez les tuiles pour découvrir la carte du vaisseau.

\subsubsection{Le Droïde}
Quand les héros arrivent dans la salle où se trouve le droïde, ce dernier les menace avec un arc électrique, rien de bien méchant mais les héros doivent le rassurer et Négocier (jet de Persuasion) avec le droïde ou le Pirater (jet de Piratage opposé à l’Intellect). Dans les deux cas, avec un succès, le droïde les aide, leur montre le journal de bord. Avec une Relance, le droïde devient un acolyte du héros qui a fait l’action. En cas d’échec le droïde refuse de les aider il se désactive. En cas d’échec critique, le droïde s’en prend à eux.

Si le droïde se désactive, un héros peut tenter de lui retirer la mémoire pour récupérer les infos avec un jet de Réparation. 

\subsubsection{Les Quartiers}
Il s’agit des quartiers de l’équipage. Pas grand-chose à y trouver, des vêtements, des effets personnels. On y trouve des terminaux qui peuvent être Piraté pour le plan du vaisseau ou pour d’autres informations. Avec un succès, ils se connectent et ont accès au contenu du PC. La carte s’ils la cherchent, sinon rien de fou, des photos, des vidéos et des films X.

\subsubsection{Les Laboratoires}
La porte du premier labo que les héros visitent est verrouillé. un jet de Piratage la déverrouille. Si le droïde est avec eux et les aide, il peut ouvrir toutes les portes. Les aventuriers trouvent dans ce labo le corps du rakghoul et un terminal. Un Piratage (malus -2, c’est un terminal sécurisé) leur donne les informations sur l’autopsie (cf. Le journal de bord~\ref{sec:pelican-jdb} p. \pageref{sec:pelican-jdb}), pas plus sur les circonstances.

Dans les autres labos, un jet de recherche donne un Medipac supplémentaire.

\subsubsection{L’Armurerie}
Dans cette pièce se trouvent des cellules énergétiques pour les armes et des armes mais enfermées dans des casiers. Un coup de blaster ouvre les casiers. On trouve alors 5x Blaster semi-automatique (2d8 (3)) et 1x Lance-Grenades (2d10 (1)), 5x Grenade (3d6).

\subsubsection{Le Pont}
La pièce est verrouillée et ne peut être ouverte avec un Piratage. Seul le droïde peut l’ouvrir. Si le droïde disparaît ou s’éteint, la salle s’ouvre. Les héros en entrant dans la pièce trouvent le corps décomposé du commandant. En recherchant dans les poches du commandant, ils trouvent un document marqué "Protocole d’Urgence" expliquant la marche à suivre en cas de problème (cf. Le journal de bord~\ref{sec:pelican-jdb} p. \pageref{sec:pelican-jdb}). Le papier contient aussi les codes d’accès au terminal.

Les héros peuvent donc accéder au journal de bord~\ref{sec:pelican-jdb} p. \pageref{sec:pelican-jdb}) avec toutes les infos. Si cela n’est déjà connu, ils savent qu’un vaisseau en état les attend dans le hangar 4-2 et il est possible d’en ouvrir la porte depuis ce terminal. S’ils sont malins, il est possible de voir ce qui les attend dans le hangar via des caméras de surveillance. Ils peuvent aussi déclencher l’autodestruction.

\begin{paperbox}{Objectifs}
Les joueurs doivent d’une manière ou d’une autre apprendre l’histoire des rakghoul et du Talisman. Donc soit par le droïde, soit par le pont. Une fois qu’une des deux étapes est passée, l’autre n’est pas obligatoire. Donc peu importe qu’ils déglinguent le droïde.
\end{paperbox}

\subsection{Le Hangard 4-2}

\noindent\includegraphics[width=\linewidth]{_img/dos-au-muur/rakghoul-amblyope.png}

Situé au pont inférieur, deuxième hangar. C’est l’étape finale et le boss de fin du scénario.

Le droïde ou un passage au poste de commande déverrouille la porte du hangar. \'A l’intérieur c’est une autre histoire, une créature de 2m de haut manifestement un rakghoul aux stéroïdes, un rakghoul Amblyope~\ref{sec:rakghoul-amblyope} p. \pageref{sec:rakghoul-amblyope}).

Selon vos héros et leur état, accompagnez le joker avec quelques rakghouls en extra. Les héros avec Sens de Force ressentent une forte présence du coté obscur autour de la créature.\\

Une fois le combat terminé, il ne leur reste plus qu’à prendre le vaisseau (Cargo léger YZ-775). Ouf, les clefs sont sur le contact. L’armement du vaisseau est endommagé et ne fonctionne pas mais le reste est intact.

\subsection{To be continue\ldots}
Une fois en vol, vos héros reçoivent une communication d’origine inconnue, le visage d’une femme apparaît

\begin{quote}
	Tinon \emph{(prononcer Taïnon)} c’est toi ? Que se passe t’il ou va tu ?
\end{quote}

Rideau, suite au prochain épisode\ldots

%Tinon Dystra
%http://www.starwars-holonet.com/encyclopedie/technologie-talisman-muur.html

\clearpage
\section{Pelican (iA-1701)}

Quelques informations sur le Pelican. Il s’agit d’un croiseur de classe \textbf{Venator}. 1 137 m de long, 7400 hommes d’équipage, capable d’hyper-propulsion.

Le Pelican appartient à Industrial Automaton qui s’en sert comme vaisseau scientifique ultra-sécurisé. Ils y hébergent des projets top secret et aux limites de la loi.

\subsection{Journal de bord}
\label{sec:pelican-jdb}
En relation étroite avec l’Empire, Industrial Automation envoie le Pelican à la recherche d’un ancien artéfact Sith, le Talisman de Muur. Le Pelican a fini par trouver une trace du Talisman sur \href{http://fr.starwars.wikia.com/wiki/Taris}{Taris}. En fouillant les bas fonds de la planète, l’équipe de chercheurs trouve les vestiges d'un ancien temple Sith. Malheureusement ce dernier est infesté de Rakghouls. L’unité de sécurité qui accompagne les chercheurs parvient à se débarrasser des quelques créatures mais plusieurs hommes sont blessés pendant le combat. Dans les ruines du temple le Talisman n’est plus là mais des indices tendent à penser qu’il a été transporté il y a très longtemps vers une planète éloignée mais les scientifiques n'ont pas eu le temps d'en apprendre plus. Ils décident alors de ramener les corps des Rakghouls à bord du Pelican et de retourner sur Gaulus, une planète rocailleuse aux confins de la bordure extérieure.

En chemin vers Jebble les scientifiques étudient le corps du Rakghoul et apprennent qu’il s’agit d’une maladie artificielle créée il y a des millénaires par Karness Muur. La maladie se transmet par griffure ou morsure. Cette maladie est étroitement liée à au coté Obscur de la Force et il semble que les personnes sensibles à la force ne puissent être contaminées. Cependant le Rakghoul étudié a l’air d’être contaminé par une forme très basique du virus, certainement une exposition prolongé à l’artéfact.

Après une semaine de voyage, les problèmes ont commencé. Les soldats blessés lors de la rixe contre les Rakghouls sur Taris commencent à se transformer en Rakghouls à leur tour et s’en prennent aux membres de l’équipage. C’est une boucherie sans nom ! Voyant cela, le commandant du Pelican (\emph{Tycho Obrin}) déclenche et protocole d’urgence consistant à enregistrer le journal de bord sur un droïde Type R et tente de bloquer le cap du vaisseau sur l’étoile la plus proche mais la propulsion est endommagé et bien que la cap du vaisseau soit bloqué, ce dernier se contente de dériver.

\subsection{Industrial Automaton}
N’ayant plus de nouvelle du Pelican depuis son départ de Taris, IA part à sa recherche. Ils le retrouvent et envoient une équipe à son bord, mais là encore, plus aucune nouvelles. C’est alors qu’IA décide d’envoyer des mercenaires. Si le commandant a respecté le protocole, il suffira que les mercenaires ramènent le droïde\ldots

\subsection{Plan du vaisseau}
\noindent\includegraphics[width=\linewidth]{_img/dos-au-muur/venator-plan.png}

Ce n’est qu’une proposition, le plan peut changer à volonté. En annexe, vous trouverez des cases permettant de faire découvrir le vaisseau petit à petit aux héros.

\clearpage
\section{Bestiaire}

\subsection{Rakghoul}
\label{sec:rakghoul}
\noindent\includegraphics[width=\linewidth]{_img/dos-au-muur/rakghoul.png}

\subsubsection{Traits}

\begin{itemtable}[ c c c c c ]
    \textbf{Agi} & \textbf{Int} & \textbf{\^Ame} & \textbf{For} & \textbf{Vig} \\
    d8			 & d6			& d6			 & d8			& d8
\end{itemtable}
\begin{itemtable}[ l X ]
	\textbf{Allure} 	 & 6 \\
	~   				 & Vision Nocturne \\
	~   				 & Marche sur les murs \\
	\textbf{Compétences} & Combat d8, Discrétion d6
\end{itemtable}

\subsubsection{Défense}
\begin{itemtable}[ c c ]
	\textbf{Parade} 	& \textbf{Résistance} \\
	6					& 6 
\end{itemtable}

\subsubsection{Attaque}
\begin{itemtable}[ X c c ]
	~ 		& \textbf{Combat} 	& \textbf{Dégats} \\
	Griffes	& d8 				& 1d6 
\end{itemtable}

\newpage
\subsubsection{Background}
Les Rakghouls sont une espèce issue d’une maladie créée par le seigneur Sith Karness Muur. Les individus atteint par cette maladie deviennent des monstres incapables de penser par eux-mêmes. Karness peut les contrôler grâce à son talisman (Le Talisman de Muur).

La maladie se transmet par une griffure ou une morsure mais cela ne fonctionne pas avec les êtres sensibles à la Force. Karness a créé ce virus à partir du coté Obscur de la Force ce qui explique une forte présence obscure près de ces monstres.

Karness a créé plusieurs versions du virus car les premiers Rakghouls ne répondaient pas bien au contrôle de Karness. Les nouveaux sont plus réceptifs et plus fort.

Quand le Talisman de Muur a été perdu dans les bas fonds de Taris, on a peu constaté que les créatures, suite à une exposition prologée au Talisman finissaient par se transformer en Rakghouls. Mais les Rakghouls transformé de cette façon sont bestiaux, stupides et sans âme. Ils attaquent tout ce qui bouge. Ces créatures ne fonctionnent qu’à l’instinct.

\clearpage
\subsection{Rakghoul Amblyope}
\label{sec:rakghoul-amblyope}
\noindent\includegraphics[width=\linewidth]{_img/dos-au-muur/rakghoul-amblyope.png}

\subsubsection{Traits}

\begin{itemtable}[ c c c c c ]
    \textbf{Agi} & \textbf{Int} & \textbf{\^Ame} & \textbf{For} & \textbf{Vig} \\
    d4			 & d6			& d6			 & d12+2		& d10
\end{itemtable}
\begin{itemtable}[ l X ]
	\textbf{Allure} 	 & 5 \\
	\textbf{Taille} 	 & +5 \\
	~   				 & Vision de Force \\
	~					 & \'Enorme (+2 pour les jets d’attaque adverses)\\
	\textbf{Compétences} & Combat d10
\end{itemtable}

\subsubsection{Défense}
\begin{itemtable}[ c c ]
	\textbf{Parade} 	& \textbf{Résistance} \\
	5					& 12 
\end{itemtable}

\subsubsection{Attaque}
\begin{itemtable}[ X c c ]
	~ 			& \textbf{Combat} 	& \textbf{Dégats} \\
	Mains nues	& d10 				& d12+2 
\end{itemtable}

\newpage
\subsubsection{Background}
Version stéroïdée des Rakghouls standard, Amblyope est notre petit boss de niveau.

Quand les Rakghouls sont livrés à eux-mêmes et qu’ils laissent libre cours à leurs plus bas instincts, il arrive qu’un Rakghoul plus fort que les autres s’en prenne à ces petits camarades et les dévore sans scrupules. Cet afflux de Force Obscure peut le faire muter et le Rakghoul devient une espèce de gros monstre de 3m de haut, complètement aveugle mais attiré par les émanations de Force, il attaque machinalement les adversaires les plus sensibles à la Force. 

\clearpage
\subsection{Vyna Anen}
\noindent\includegraphics[width=\linewidth]{_img/dos-au-muur/vyna-anen.png}

\textbf{Race:} Sluissi

\subsubsection{Traits}

\begin{itemtable}[ c c c c c ]
    \textbf{Agi} & \textbf{Int} & \textbf{\^Ame} & \textbf{For} & \textbf{Vig} \\
    d6			 & d10			& d6			 & d4			& d6
\end{itemtable}
\begin{itemtable}[ l X ]
	\textbf{Allure} 	 & 6 \\
	\textbf{Compétences} & Intimidation d6, Persuasion d12, Réseaux d6
\end{itemtable}

\subsubsection{Défense}
\begin{itemtable}[ c c ]
	\textbf{Parade} 	& \textbf{Résistance} \\
	2					& 5 
\end{itemtable}

\newpage
\subsubsection{Background}
\noindent\includegraphics[width=\linewidth]{_img/dos-au-muur/industrial-automaton.png}

Vyna Anen est l’un des agents de liaison entre Industrial Automaton et l’Empire. Officiellement employé par IA comme secrétaire au service des "Projets Spéciaux". 

Vyna est un homme pragmatique qui fait ce que l’Empire lui demande sans poser de questions, sans scrupules ni états d’âme. C’est un fin négociateur entièrement voué à l’Empire.\\


\subsection{Droïde R4 [Joker]}


	ici les héros sont dans un navette et s'échape du Pelican qui fonce droit sur une étoile.

Une communication entrante les interpelle et leur demande des compte. Il s'agit de la résistance.

une fois l'appel terminé les héros disposent de trois choix :
 - Raméner le droide à Industrial Automaton
 - Rejoindre la résistance
 - Ou partir directement sur Gaulus au labo d'Industrial Automaton
 - voire un 4ieme de partir pour Taris

Bon l'idée c'est que dans tout les cas la prochaine étape est Taris et le temple Sith dans lequel ils trouverons un holocron qui leur dira que le Talisman est partie sur Jebble avec Céleste Morne et Pulcipher.
L'holocron parle de l'oubliette de Dreypa et de la rivalité entre Dreypa et Karness. Il ne dit pas ou se trouve l'oubliette ni le talisman. 
Pas loin se trouve les restes du campement de Pulcipher avec son journal qui explique qu'il a l'intention de partir pour Jebble à son labo pour y étudier l'artéfact. Le dernier message est coupé avant la fin mais il semble que Pulcipher et été interrompu par un duo de Jedi et ai du quitté Taris en précipitation.


L'épisode suivant se passe sur Jebble dans le labo du professeur Pulcipher. Où l'on apprend ce qu'il s'est passé pendant le voyage de retour et où l'on a des piste sur l'endroit ou se trouve l'oubliette. On entend notement parler de Céleste Morne.
Ici une idée est qu'au moment de quitté la planète pour l'étape suivante, les héros se retrouvent pris au piège par des troupes de l'empire qui on pris leur vaisseau en otage. Histoire de varier l'aventure. On peut même se mettre une petite baston spaciale.
Il faudra ensuite forcer les héros à retourner à leur QG (alliance rebelle ou empire) pour réparation et pour compte rendu.

Dans l'épisode suivant, le rebelles apprennent qu'on aurait vu le talisman sur une certaine lune de Jesaispasou et les soldats de l'empire apprenne qu'ils vont tendre un piège à l'alliance sur un lune de Jesaispasou.
Grossomerdo, Celeste se pointe et calme tout le monde, obligé de battre en retraite et de trouver un plan pour enfermer Celeste dans l'oubliette de Dreypa ou pour lui virer le Talisman avant d'enfermer ce dernier dans l'oubliette. Ou se trouve l'oubliette ? Quel est le plan ?
	\section{Laboratoire de Pulsipher}
L'épisode suivant se passe sur Jebble dans le labo du professeur Pulcipher. Où l'on apprend ce qu'il s'est passé pendant le voyage de retour. On trouve aussi des information sur l'oubliette de Dreypa et l'on a des piste sur l'endroit où elle se trouve. On entend notement parler de Céleste Morne.
Ici une idée est qu'au moment de quitté la planète pour l'étape suivante, les héros se retrouvent pris au piège par des troupes de l'empire qui on pris leur vaisseau en otage. Histoire de varier l'aventure. On peut même se mettre une petite baston spaciale.
Il faudra ensuite forcer les héros à retourner à leur QG (alliance rebelle ou empire) pour réparation et pour compte rendu.

Dans l'épisode suivant, le rebelles apprennent qu'on aurait vu le talisman sur une certaine lune de Jesaispasou et les soldats de l'empire apprenne qu'ils vont tendre un piège à l'alliance sur un lune de Jesaispasou.
Grossomerdo, Celeste se pointe et calme tout le monde, obligé de battre en retraite et de trouver un plan pour enfermer Celeste dans l'oubliette de Dreypa ou pour lui virer le Talisman avant d'enfermer ce dernier dans l'oubliette. Ou se trouve l'oubliette ? Quel est le plan ?
	\section{C’est dans la boite}


\subsection{Au rapport !}
Nos héros quittant Jebble son contacté par leur faction.

\subsubsection{Empire}
\begin{quotebox}
    \nameref{sec:garan-keggle}: Au rapport ! Comment avance vos recherches sur l’artefact ?
\end{quotebox}
Les joueurs racontent\ldots
\begin{quotebox}
    \nameref{sec:garan-keggle}: La "Boite de Jebble"\ldots Ca ne me dit rien ! Mais allez voir \nameref{sec:fane-peturri} sur \textbf{Muunilinst}, c’est un historien de l’Empire, ami de notre Empereur, il pourra certainement vous aider. Je vais le prévenir de votre visite et je vous transfère ses coordonnées.
\end{quotebox}

Les héros se rendent donc sur \textbf{Muunilinst} (normalement) et rencontrent \nameref{sec:fane-peturri}. Ce dernier les invite chez lui, leur offre le thé et les reçoient bien.
\begin{quotebox}
    \nameref{sec:fane-peturri}: Garan m’a prévenu de votre arrivée mais il ne m’a pas dit quel était le but de votre visite ? En quoi puis-je aider l’empire ?
\end{quotebox}
Les héros devraient donc lui parler de la fameuse boite de Jebble qui est en réalité \nameref{sec:oubliette-de-dreypa}.

\begin{quotebox}
    \nameref{sec:fane-peturri}: Il me semblait bien qu’il y avait un rapport entre l’oubliette et cette Boite ! Il y a quelques temps, suite à une réquisition d’oeuvres d’art pour le compte de l’empire, je suis tombé sur un enregistrement holo qui parlait de cette boite. D’après ce que disant l’enregistrement, mais il date maintenant de plusieurs mois, la \textbf{Boite de Jebble} se trouve sur le \nameref{sec:uhumele} un cargo qui verse dans le commerce et la contrebande de babiole diverses. Son capitaine, \nameref{sec:schurk-heren} est un Yarkora qui se méfie de l’Empire. Il vous faudra sans doute éviter de mentionner que vous travaillez pour l’Empereur.

    Le dernier port d’attache que je lui connais est \textbf{Pizkoss}, il est certainement en train d’essayé de refourguer sa marchandise !
\end{quotebox}

\subsubsection{Rebelion}
\begin{quotebox}
    \nameref{sec:lindi-dangon}: Bonjour, je viens aux nouvelles, comment se passe la recherche de l’artéfact ? Vous avez besoin de quelque chose ?
\end{quotebox}

Les joueurs racontent\ldots

\begin{quotebox}
    \nameref{sec:lindi-dangon}: La \textbf{Boite de Jebble}, ça me dit quelque chose, attendez une seconde\ldots 

    \textit{Lindi disparait de l’écran un instant puis revient}

    \nameref{sec:lindi-dangon}: Effectivement, un Jedi en parle dans un de ses rapports. \nameref{sec:dass-jennir}, il n’est pas très précis dans son rapport, mais vous devriez aller le voir. Il est en retraite sur \textbf{Muunilinst}. Je vous fais suivre les coordonnées où vous le trouverez, soyez respectueux et diplomates ! Rappelez vous que c’est un Jedi !
\end{quotebox}

Les héros se rendent donc sur \textbf{Muunilinst} (normalement) pour rencontrer \nameref{sec:dass-jennir}. Dass Jennir se montre tout d’abord froid et distant
\begin{quotebox}
    \nameref{sec:dass-jennir}: C’est pour quoi ? Si c’est encore pour réparer votre moissonneuse, revenez plus tard, je suis occupé là.
\end{quotebox}

Aux héros de se montrer diplomate pour l’amadouer ! Quand c’est fait. Ils lui parle de la Boite de Jebble. En entendant ce nom, le visage de Dass Jennir marque un sentiment de souffrance. C’est manifestement un souvenir douloureux
\begin{quotebox}
    \nameref{sec:dass-jennir}: Oui, je me souviens de ça ! Je ne l’ai pas vu personnellement mais elle faisait partie de l’inventaire des objets de collection pillé par l’\nameref{sec:uhumele} lors d’une mission qui a mal tournée. L’équipage de vaisseau à gardé tout un tas de babioles afin de se rembourser l’échec de la mission. La boite devait se trouver parmis les objets.

    \textit{\ldots bla bla les héros essaient d’en savoir plus \ldots}

    \nameref{sec:dass-jennir}: Je suppose que \nameref{sec:schurk-heren}, le capitaine doit être en train de refourguer sa marchandise quelque part sur \textbf{Pizkoss} !
\end{quotebox}


\subsection{Uhumele}\label{sec:uhumele}
%   Ici une idée est qu'au moment de quitté la planète pour l'étape suivante, les héros se retrouvent pris au piège par des troupes de l'empire qui on pris leur vaisseau en otage. Histoire de varier l'aventure. On peut même se mettre une petite baston spaciale.

% La suite c’est qu'il faut retrouver la boite avant que quelqu'un d'autre ne mette la main dessus (l'Empire ou un rival de Palpatine). Mais c'est raté, c'est les méchant d'en face qui récupèrent en premier et qui l'ouvre.

% scénario suivant et final, les méchant tendent un piège aux héros grace à Céleste mais les gentils parviennent à libérer Celeste et à remettre l'amulette dans l'oubliette. Cette dernière est récupérer par la faction des héros et conservée à l'abris.
	\section{Personnages}
Les personnages avec un ‘ \textbf{*} ’ sont des Jokers, ils possèdent un fiche de perso jouable. 

Par ordre d'apparition :

\begin{figure}[h!]
    \centering
    \includegraphics[width=\linewidth]{_img/dos-au-muur/industrial-automaton.png}
    \caption{Industrial Automaton}
\end{figure}

\newpage
\subsection{Vyna Anen} \label{sec:vyna-anen}
\noindent\includegraphics[width=\linewidth]{_img/dos-au-muur/vyna-anen.png}
\textbf{Race:} Sluissi

\subsubsection{Background}

Vyna Anen est l’un des agents de liaison entre Industrial Automaton et l’Empire. Officiellement employé par IA comme secrétaire au service des "Projets Spéciaux". 

Vyna est un homme pragmatique qui fait ce que l’Empire lui demande sans poser de questions, sans scrupules ni états d’âme. C’est un fin négociateur entièrement voué à l’Empire.

\subsubsection{Traits}

\begin{itemtable}[ c c c c c ]
    \textbf{Agi} & \textbf{Int} & \textbf{\^Ame} & \textbf{For} & \textbf{Vig} \\
    d6           & d10          & d6             & d4           & d6
\end{itemtable}
\begin{itemtable}[ l X ]
    \textbf{Allure}      & 6 \\
    \textbf{Compétences} & Intimidation d6, Persuasion d12, Réseaux d6
\end{itemtable}

\subsubsection{Défense}
\begin{itemtable}[ c c ]
    \textbf{Parade}     & \textbf{Résistance} \\
    2                   & 5 
\end{itemtable} 

\clearpage
\subsection{R4-3D*} \label{sec:r4-3d}

\clearpage
\subsection{Tinon Dystra} \label{sec:tinon-dystra}
\begin{figure}[h!]
    \centering
    \includegraphics[height=250pt]{_img/dos-au-muur/tinon-dystra.png}
\end{figure}

\subsubsection{Background}
Ce personnage est mort quand débute le scénario mais il garde son importance car il est le lien entre les joueurs et l'alliance Rebelle.

Tinon est un membre de l'alliance rebelle envoyé en mission d'infiltration sur un vaisson de Industrial Automaton (Le \nameref{sec:pelican}) afin de découvrir ce qu'il s'y trame. Mais sa mission à mal tourné et il n'a plus donné aucune nouvelle. 

En fait il s'est infiltré à bord du Pelican alors que la maladie des \nameref{sec:rakghoul} était en train de s'y répandre. A bord du Pelican il s'est fait infecté et s'est transformé en Rakghoul lui-même. Son vaisseau, le \nameref{sec:nimbus} est resté à quai sur le Pelican.

Tinon avait une relation avec \nameref{sec:lindi-dangon}.

\newpage
\subsection{Lindi Dangon} \label{sec:lindi-dangon}
\begin{figure}[h!]
    \centering
    \includegraphics[height=250pt]{_img/dos-au-muur/lindi-dangon.png}
\end{figure}
\subsubsection{Background}
Lindi Dangon commande l'une des cellule de résistance dans la zone de Taris. C'est elle qui a ordonné la mission durant laquelle \nameref{sec:tinon-dystra} à disparut. Elle se sent d'autant plus coupable que Tinon était son amant et depuis sa disparition elle n'a de sesse de la retrouver. Elle garde espoir tant qu'elle n'a pas de preuve de sa mort.

\subsubsection{Traits}

\begin{itemtable}[ c c c c c ]
    \textbf{Agi} & \textbf{Int} & \textbf{\^Ame} & \textbf{For} & \textbf{Vig} \\
    d6           & d10          & d8             & d4           & d4           
\end{itemtable}
\begin{itemtable}[ l X ]
    \textbf{Allure}      & 6 \\
    \textbf{Compétences} & Intimidation d8, Persuasion d8, Réseaux d10, Tir d10, Combat d4 \\
    \textdb{Atouts}      & Commandement
\end{itemtable}

\subsubsection{Défense}
\begin{itemtable}[ c c ]
    \textbf{Parade}     & \textbf{Résistance} \\
    5                   & 3 
\end{itemtable}

\newpage
\subsection{Garan Keggle}  \label{sec:garan-keggle}
\begin{figure}[h!]
    \centering
    \includegraphics[height=200pt]{_img/dos-au-muur/garan-keggle.png}
\end{figure}
\vspace{-1\baselineskip}
\subsubsection{Background}
Officier supérieur dans l'armée de l'empire, Garan Keggle dirige les recherches d'artéfacts Sith pour le compte de l'empereur et son second Dark Vador. Suite à des informations sur le Talisman de Muur, il a dépéché un vaisseau de l'Insdustrial Automaton pour récupérer l'artéfact sur Taris sous couvert de projet scientifique.

Son contact chez IA est \nameref{sec:vyna-anen}, de manière générale Garan n'a que très peu de contact avec ses sous-traitant pour éviter que les ennuis remontent jsuqu'à l'empire. 

Garan est un home déterminé et sans pitié, initié au coté obscur de la Force, il attend la première occasion pour se débarrasser de ses rivaux et pouvoir prétendre au plus vite à la place d'apprenti et plus tard de seigneur Sith.

\subsubsection{Traits}
\begin{itemtable}[ c c c c c ]
    \textbf{Agi} & \textbf{Int} & \textbf{\^Ame} & \textbf{For} & \textbf{Vig} \\
    d6           & d10          & d10            & d6           & d8           
\end{itemtable}
\begin{itemtable}[ l X ]
    \textbf{Allure}      & 6 \\
    \textbf{Compétences} & Intimidation d10, Tir d10, Combat d8, Maitrise Force d8, Perception d6 \\
    \textdb{Atouts}      & Commandement, Grande aura de commandement
\end{itemtable}

\subsubsection{Défense}
\begin{itemtable}[ c c ]
    \textbf{Parade}     & \textbf{Résistance} \\
    6                   & 6 
\end{itemtable}

\newpage
\subsection{Gil Harend}  \label{sec:gil-harend}
\begin{figure}[h!]
    \centering
    \includegraphics[height=200pt]{_img/dos-au-muur/gil-harend.png}
\end{figure}
\subsubsection{Background}
Double background selon la voie choisie.

Gil Harend est un père qui essai tant bien que mal d'élever sa fille de 16 ans, Abygaelle, dans les bas fonds de Taris. Depuis la disparition se sa femme Karie, la vie est très difficile. De plus les contrebandiers font régulièrement des razzia dans leur village pour kidnapper les jeunes filles et les revendre comme esclaves. Il y a deux jours, c'est Abygaelle qui a été enlevée. Gil fera tout pour la retrouver.\\

Gil Harend est un contrebandier qui opère dans les bas fonds de Taris. Avec ses 3 accolytes Gil fait des petits boulots plus ou moins légaux mais ce qui rapporte le plus c'est la revente de jeunes esclaves. Il y a 2 jours, lui est son équipe on kidnappés une jeune fille de 16 ans dans un village des bas fonds.

\subsubsection{Traits}
\begin{itemtable}[ c c c c c ]
    \textbf{Agi} & \textbf{Int} & \textbf{\^Ame} & \textbf{For} & \textbf{Vig} \\
    d4           & d8           & d4             & d8           & d8           
\end{itemtable}
\begin{itemtable}[ l X ]
    \textbf{Allure}      & 6 \\
    \textbf{Compétences} & Tir d6, Combat d4, Persuasion d6
\end{itemtable}

\subsubsection{Défense}
\begin{itemtable}[ c c ]
    \textbf{Parade}     & \textbf{Résistance} \\
    4                   & 6 
\end{itemtable}

\newpage
\subsection{Pulsipher} \label{sec:pulsipher}
\begin{figure}[h!]
    \centering
    \includegraphics[height=200pt]{_img/dos-au-muur/pulsipher.png}
\end{figure}

\subsubsection{Background}
Pulsipher est un Mandalorien qui vécut plus de 3000 ans avant l'avènement de l'empire. Scientifique Néo-Croisé, il s'intéressait à la Force et tout ce qui s'y rattache. Il était persuadé que le secret de la Force lui permettrait de mettre fin à la guerre. 

C'est lui qui découvrit dans les bas-fonds de Taris et qui le ramena sur Jebble.

\subsubsection{Traits}
\begin{itemtable}[ c c c c c ]
    \textbf{Agi} & \textbf{Int} & \textbf{\^Ame} & \textbf{For} & \textbf{Vig} \\
    d4           & d12          & d4             & d8           & d6           
\end{itemtable}
\begin{itemtable}[ l X ]
    \textbf{Allure}      & 6 \\
    \textbf{Compétences} & Tir d8, Combat d8, Persuasion d6, Connaissance d12
\end{itemtable}

\subsubsection{Défense}
\begin{itemtable}[ c c ]
    \textbf{Parade}     & \textbf{Résistance} \\
    6                   & 8 
\end{itemtable}

\newpage
\subsection{Lucy} \label{sec:lucy-pher}

	\section{Bestiaire}

\subsection{Storm Trooper} \label{sec:storm-trooper}
\begin{figure}[h!]
    \centering
    \includegraphics[height=200pt]{_img/dos-au-muur/stormtrooper.png}
\end{figure}
\paragraph{Background}
Soldats dévoué de l’empire. Certain sont des clones restant de la guerre des clones d’autres non. Ils sont entrainés au combat, équipé d’une bonne armure et armé de Fusil Blaster efficaces.

\paragraph{Traits}

\begin{itemtable}[ c c c c c ]
    \textbf{Agi} & \textbf{Int} & \textbf{\^Ame} & \textbf{For} & \textbf{Vig} \\
    d4           & d6           & d4             & d8           & d8
\end{itemtable}
\begin{itemtable}[ l X ]
    \textbf{Allure}      & 6 \\
    \textbf{Compétences} & Combat d10, Tir d10
\end{itemtable}

\paragraph{Défense}
\begin{itemtable}[ c c ]
    \textbf{Parade}     & \textbf{Résistance} \\
    7                   & 6 (+4)
\end{itemtable}

\paragraph{Attaque}
\begin{itemtable}[ X c c ]
    ~              & \textbf{Combat}   & \textbf{Dégats} \\
    Fusil Blaster  & -                 & 2d8 (3)
\end{itemtable}


\newpage


	\onecolumn
	\nocite{*}
	\printbibliography
\end{document}