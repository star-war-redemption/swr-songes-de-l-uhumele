\section{Les bas fonds de Taris}

Taris est une planète urbaine très polluée de la Bordure Extérieure. Divisée en trois grandes parties en fonction des étages, la Ville Haute, la ville médiane et la ville basse aussi appelée "bas fonds", elle abrite une population gigantesque. Grand pôle économique dans la galaxie, notamment grâce à l’implantation des industries Lhosan, elle joignit la République Galactique peu avant les Guerres Mandaloriennes. 

\subsection{Résumé des épisodes précédents}
Nos héros se retrouvent donc dans un Cargo léger YZ-775 sans canons fonctionnel et un message entrant d’une inconnue qui demande à parler à Tinon.

Ils ont aussi apris qu’Industrial Automaton sous couvert d’expérimentations de son nouveau modèle, fait des recherches sur un très ancien artefact Sith en provenance de \textbf{Taris}. De plus, s’ils ont réussi à récupérer le droïde, Vyna Anen les attend sur l’avant-poste commercial de \textbf{l’anneau de Kafrene} au Starlord Café. Enfin, ils savent aussi que le vaisseau se dirigeait vers \textbf{Gaulus} pour rejoindre un laboratoire de l’empire où il devait finir des recherches sur l’artéfact.

Plusieurs directions s’offrent alors à nos héros.

\subsection{Les bon, les brutes ou les truants}
\begin{quotebox}
    Tinon \emph{(prononcer Taïnon)} c’est toi ? Que se passe t’il ou va tu ?
\end{quotebox}
Le visage d’une jeune femme apparait sur l’holocomm. Les héros ont le choix de répondre ou non.

Ce point de l’aventure est crutial pour nos héros car ils vont choisir (en connaissance de cause ou non) l’orientation de leurs personnages. En effet plusieurs cas de figures se présentent :

\begin{description}
    \item[\nameref{sec:les-rebels}] Répondre au message et suivre les consignes des Rebels (page~\pageref{sec:les-rebels}).
    \item[\nameref{sec:retour-du-droide}] Ramener le droïde à Vyna Anen (page~\pageref{sec:retour-du-droide}).
    \item[\nameref{sec:refus-d-obtemperer}] Partir pour Taris directement (page~\pageref{sec:refus-d-obtemperer}).
    \item[\nameref{sec:l-empire}] Prendre la direction de Gaulus (page~\pageref{sec:l-empire}).
\end{description}

Tous ses choix amèneront, de toute façon, les héros sur Taris mais le chemin sera différent et par là même leur orientation aussi. Je vais tacher de décrire les choix possibles sachant que rien n’est écrit dans le marbre et vous pouvez très bien les forcer suivre une route en particulier. C'est un peu le jeu de piste, les choix s'entre-croisent alors faut suivre. Mais une fois de plus rien n'est définitif, j'ai juste essayé de prévoir le plus de cas possible pour limiter l'improvisation. Les MJs expérimentés n'auront sans doute pas besoin de tout ça.


\subsubsection{Retour du droïde} \label{sec:retour-du-droide}

Autre choix possible pour les héros, ces derniers, avident de crédits, veulent allez rendre le droïde et récupérer le solde promis pour leur mission. Il retournent donc sur l'avant-poste commercial dans \textbf{l'anneau de Kafrène} ou les attend \nameref{sec:vyna-anen}.

On laisse les héros discuter un peu et réclamer leur due, Vyna acquiesce avec le sourire, leur donne les crédits mais au moment de quitter la pièce, 4 ou 5 \textit{(doser en fonction du niveau de vos joueurs)} Stormtroopers leur barre la route. Attention, à moins que les héros est anticipé, le premier tour de combat sera pour dégainer.

\begin{quotebox}
    Vous pensiez réellement qu'avec tout ce que vous savez, nous allons vous laisser repartir tranquillement ?\\
    Commencez par me rendre les crédits.
\end{quotebox}

Les héros peuvent tenter de négocier avec Vyna, ce dernier acceptera de les laissé repartir vivant (mais sans crédit) à condition qu'ils retourne sur \nameref{sec:taris} et recupèrent les informations que les scientifiques de Industrial Automaton n'ont pas eu le temps de récupérer. Ils peuvent tenter un jet de Négociation pour garder les crédits.\\

Si les héros entrent en combat, et qu'ils perdent, on ne les tue pas mais Vyna leur impose de retourner sur Taris et pas de négociation possible. S'ils gagnent, deux cas de figure possible :
\begin{rebelist}
    \item \textbf{Ils tuent Vyna} Un homme, dathomirien, manifestement important en uniforme de l'empire entre dans la pièce en applaudissant. Pausément, il félicite les joueurs et   leur expliquent qu'il ont réussi le test avec brillo. Vyna était devenu génant pour l'empire et pour l'Odre. Si l'un des héros tente quoi que ce soit il se retrouve désarmé et collé au mur avant d'avoir eu le temps de comprendre. On déccroche sur \nameref{sec:l-empire}.

    \item \textbf{Ils laisse Vyna en vie} Une femme (celle qui est apparue sur le holocom du vaisseau) entre dans la pièce avec deux hommes. Les deux hommes emportent Vyna avec eux, en faisant comme s'il était ivre. Et la femme leur fait signe de la suivre vite, elle leur expliquera plus tard. Ils prennent le vaisseau et partent pour la cache des Rebels. On déccroche sur \nameref{sec:les-rebels} dans la version bien accueilli.
\end{rebelist}

Le \nameref{sec:nimbus} n'est réparé qu'à leur demande, le prix est de 40 000 \crg. Négociable à 30 000, -10k si relance.


\subsubsection{Les rebels} \label{sec:les-rebels}

\begin{quotebox}
    Tinon \emph{(prononcer Taïnon)} c’est toi ? Que se passe t’il ou va tu ?
\end{quotebox}

Le visage d’une jeune femme apparait sur l’holocomm et les héros choisissent de répondre.

\begin{quotebox}
    Mais qui êtes vous et où et Tinon ?
\end{quotebox}
Les joueurs choisissent de raconter leur histoire, ou de mentir, à voir ce qu'ils préfèrent. Lindi leur demande de la rejoindre et leur donne les coordonnées d'une cache de la Rébellion. Si les joueurs ont menti, il sont accueilli avec suspicion, menacé et tout ce qui va avec, sinon ils sont accueillit normalement avec juste un peu de méfiance. \textbf{Lindi Dangon} les invite à la suivre et les interroge jusqu'à ce qu'ils disent la vérité (On part du principe que s'ils persistent à mentir, on bifurque vers l'option \nameref{sec:refus-d-obtemperer}).

Lindi leur explique alors que la résistance à eu vent des recherches mené par l'Industrial Automaton ainsi que des problèmes sur le Pelican. \textbf{Tinon Dystra} avait été envoyé pour tenté de s'infiltrer à bord du Pelican. Mais qu'il n'avait plus donné de nouvelles depuis son arrimage au vaisseau.

Un jet de \textbf{Perception} ou l'utilisation de \textbf{Sens de Force} fera apparaître les sentiments de Lindi envers Tinon et apprendra à nos héros que ces derniers étaient amants.\\

Une fois la discution terminé, Lindi demande de l'aide aux héros, qu'ils reprenne le travail de Tinon. La première piste à explorer étant \ldots \nameref{sec:taris}.

Le \nameref{sec:nimbus} est réparé gratuitement pendant que les héros discutent.


\subsubsection{Refus d’obtempérer} \label{sec:refus-d-obtemperer}

Les héros veulent se la jouer et décident de partir direct sur Taris. Mais ils ne savent même pas trop se qu'ils y cherche. L'hyper-espace est récalcitrant et ne veut pas fonctionner immédiatement. Le temps de regarder, un croiseur sort d'hyper-espace et les arraisonne sans poser de question. Des soldats de l'alliance (trop pour être combattu) forcent la cloison et entre dans le cargo, suivi par la femme vu précédemment sur l'holocomm.

\begin{quotebox}
    Où pensiez vous partir comme ça avec ce cargo ? \\
    \emph{se retournant vers les soldats}, jetez moi ça en cellule on les interrogera là bas.
\end{quotebox}

Les héros sont transporté jusqu'à une cache de la Rébellion et jeté en cellule. Au bout de quelques temps, la femme précédement rencontré vient les interroger. Ils ont le choix de répondre la vérité ou de mentir. S'ils répondent la vérité, on déccroche sur l'option correspondante dans \nameref{sec:les-rebels}. Sinon ils sont laissé en cellule \ldots

Mais dans la nuit, un mystérieux inconnu s'approche de la cellule sans bruit

\begin{quotebox}
    Je suis envoyé par Vyna Anen, si vous voulez vivre, venez avec moi et ne faites pas un bruit; \\
\end{quotebox}

L'inconnu ouvre leur cellule et les emmène au Cargo. Sur le chemin, leur demander des jets de \textbf{Discrétion}. En cas d'échec deux soldats débarquent. Dans le hangar où se trouve le Cargo, 4 soldats montent la garde. Soient les héros passent en force et débute un combat puis vont ouvrir les portes du hangar. Soit il la jou furtif, rentre dans le vaisseau et tire sur les porte pour sortir du hangar.



\subsubsection{L’empire} \label{sec:l-empire}

\subsection{Taris} \label{sec:taris}

\subsection{Nimbus} \label{sec:nimbus}
Cargo léger YZ-775

-----


Bon l'idée c'est que dans tout les cas la prochaine étape est Taris et le temple Sith dans lequel ils trouverons un holocron qui leur dira que le Talisman est partie sur Jebble avec Céleste Morne et Pulcipher.
L'holocron parle de l'oubliette de Dreypa et de la rivalité entre Dreypa et Karness. Il ne dit pas ou se trouve l'oubliette ni le talisman. 
Pas loin se trouve les restes du campement de Pulcipher avec son journal qui explique qu'il a l'intention de partir pour Gaulus à son labo pour y étudier l'artéfact. Le dernier message est coupé avant la fin mais il semble que Pulcipher et été interrompu par un duo de Jedi et ai du quitté Taris en précipitation.


L'épisode suivant se passe sur Jebble dans le labo du professeur Pulcipher. Où l'on apprend ce qu'il s'est passé pendant le voyage de retour et où l'on a des piste sur l'endroit ou se trouve l'oubliette. On entend notement parler de Céleste Morne.
Ici une idée est qu'au moment de quitté la planète pour l'étape suivante, les héros se retrouvent pris au piège par des troupes de l'empire qui on pris leur vaisseau en otage. Histoire de varier l'aventure. On peut même se mettre une petite baston spaciale.
Il faudra ensuite forcer les héros à retourner à leur QG (alliance rebelle ou empire) pour réparation et pour compte rendu.

Dans l'épisode suivant, le rebelles apprennent qu'on aurait vu le talisman sur une certaine lune de Jesaispasou et les soldats de l'empire apprenne qu'ils vont tendre un piège à l'alliance sur un lune de Jesaispasou.
Grossomerdo, Celeste se pointe et calme tout le monde, obligé de battre en retraite et de trouver un plan pour enfermer Celeste dans l'oubliette de Dreypa ou pour lui virer le Talisman avant d'enfermer ce dernier dans l'oubliette. Ou se trouve l'oubliette ? Quel est le plan ?